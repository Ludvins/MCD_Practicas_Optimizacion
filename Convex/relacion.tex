\documentclass[12pt]{article}
\usepackage[utf8]{inputenc}
\usepackage[T1]{fontenc}
\usepackage{grffile}
\usepackage{longtable}
\usepackage{wrapfig}
\usepackage{rotating}
\usepackage[normalem]{ulem}
\usepackage{amsmath}
\usepackage{textcomp}
\usepackage{amssymb}
\usepackage{capt-of}
\usepackage{hyperref}
\hypersetup{colorlinks=true, linkcolor=magenta, citecolor=cyan}
\setlength{\parindent}{0in}
\usepackage[margin=1in]{geometry}
\usepackage[english]{babel}
\usepackage{mathtools}
\usepackage{palatino}
\usepackage{fancyhdr}
\usepackage{sectsty}
\usepackage{engord}
\usepackage{parskip}
\usepackage{minted}
\usepackage{graphicx}
\usepackage{subcaption}
\usepackage{setspace}
%\usepackage[center]{caption}
\usepackage{placeins}
\usepackage{color}
\usepackage{amsmath}
\usepackage{bm}
\usepackage{todonotes}
\usepackage{pdfpages}
% \titlespacing*{\subsection}{0pt}{5.5ex}{3.3ex}
%\titlespacing*{\section}{0pt}{5.5ex}{1ex}
\usepackage{natbib}

\author{Luis Antonio Ortega Andrés\\Antonio Coín Castro}
\date{\today}
\title{Convex Optimization \\ \Large Exercises}
\hypersetup{
 pdfauthor={Luis Antonio Ortega Andrés, Antonio Coín Castro},
 pdftitle={},
 pdfkeywords={},
 pdfsubject={},
 pdflang={English}
 }

 \newcommand*{\QED}{\null\nobreak\hfill\ensuremath{\square}}%

\begin{document}

\maketitle


\textit{Luis Antonio: 50\%}, \ \textit{Antonio: 50\%}\\
\textbf{Exercise 1. }\emph{Show that if \( S \) is an open set, its complement \( S^c \) is closed, and viceversa.}

\emph{Solution.} Suppose $S$ is an open set. We will prove that $S^c$ is closed by showing that it coincides with its closure $\operatorname{cl}(S^c)$. On the one hand, we have by definition that \( S^c \subset \operatorname{cl}({S^c})\), since
\[
     \operatorname{cl}(S^c) =  \{x : \ \forall \delta\,  B(x, \delta) \cap S^c \neq \emptyset \}.
\]

On the other hand, let \( x \in \operatorname{cl}({S^c}) \) and $\delta>0$. Then, we have:
\begin{equation}
  \label{eq:1}
     B(x, \delta) \cap S^c \neq \emptyset \implies B(x, \delta) \not \subset S.
\end{equation}

But since \( S \) is open, it holds that \( \forall x \in S \ \exists \delta > 0 \ \)  such that \( \ B(x, \delta) \subset S\). Therefore, as $\delta$ was arbitrary in \eqref{eq:1}, necessarily $x\notin S$, and so $x\in S^c$.  As $x$ was also arbitrary, we have shown that $cl(S^c)\subset S^c$, and hence that the complement of an open set is closed, as desired. The converse is obvious by taking complements.\QED\\

\textit{Luis Antonio: 50\%}, \ \textit{Antonio: 50\%}\\
\textbf{Exercise 2. }\emph{If \( S_1, S_2 \) are convex subsets, prove that \( S_1 \cap S_2, S_1 + S_2 \) and \( S_1 - S_2 \) are also convex sets.}

\emph{Solution.} We will begin with \( S_1 \cap S_2 \). Let \( x, x' \in S_1 \cap S_2 \) and \( \lambda \in [0,1] \). By definition of $S_1\cap S_2$ we have $x,x'\in S_1$ and $x,x'\in S_2$, and by convexity it holds that
\[
     \lambda x + (1-\lambda)x' \in S_1 \quad \text{ and } \quad  \lambda x + (1-\lambda)x' \in S_2
\]
Therefore, \(  \lambda x + (1-\lambda)x' \in S_1 \cap S_2 \), proving that the set is indeed convex.

Consider now \( x, x' \in S_1 + S_2 \). We can decompose them as \(x = x_1 + x_2 \) and \( x' = x_1' + x_2' \), where $x_1, x_1'\in S_1$ and $x_2, x_2'\in S_2$. If $\lambda\in[0,1]$, using the convexity of $S_1$ and $S_2$ we have:
\begin{align*}
   \lambda x + (1-\lambda)x' &= \lambda x_1 + \lambda x_2 + x_1' + x_2' - \lambda x_1' - \lambda x_2'\\
   &= \underbrace{\lambda x_1 + (1-\lambda) x_1'}_{\in S_1} + \underbrace{\lambda x_2 + (1-\lambda) x_2'}_{\in S_2} \in S_1 + S_2.
\end{align*}

Lastly, the set \( S_1 - S_2 \) is shown to be convex using the same reasoning as we did for $S_1+S_2$.\QED\\

\textit{Luis Antonio: 50\%}, \ \textit{Antonio: 50\%}\\
\textbf{Exercise 3. }\emph{If \( f: S \to \mathbb{R} \) is a convex function on the convex set \( S \), the set \( \{x : \ x \text{ is a minimum of f} \} \) is a convex set.}

\emph{Solution.} Let \( M(f) = \{x \in S: \ x \text{ is a minimum of f} \} \) and suppose it is non-empty (otherwise there is nothing to prove, since the empty set is convex). Consider \( x, x' \in M(f) \) and \( \lambda \in [0,1] \). Now, let  $c\in \mathbb R$ be the minimal value attained by $f$, so that $f(x)=f(x')=c$. Using this fact and the convexity of $f$, we have:
\[
    f(\lambda x + (1-\lambda)x') \leq \lambda f(x) + (1-\lambda)f(x') = \lambda c + (1-\lambda)c = c,
\]
but since $c$ is the minimal value of $f$, we also have $f(\lambda x + (1-\lambda)x') \geq c$, so necessarily $f(\lambda x + (1-\lambda)x')=c$, and hence $\lambda x + (1-\lambda)x' \in M(f)$. As a result, \( M(f) \) is a convex subset of \( S \).\QED\\

\textit{Luis Antonio: 50\%}, \ \textit{Antonio: 50\%}\\
\textbf{Exercise 4. }\emph{Let \( f : S \subset \mathbb{R}^d  \to \mathbb{R}\) be a convex function on the convex set \( S \), and suppose we extend it to an \( \hat{f}:\mathbb{R}^d \to \mathbb{R}\) as  }
\[
    \hat{f}(x) = \begin{cases}
        f(x), &\text{ if } x \in S,\\
        +\infty, &\text{ if } x \notin S.
    \end{cases}
\]
\emph{Show that \( \hat{f} \) is a convex function on \( \mathbb{R}^d \) }.

\emph{Solution}. Let \( x, x' \in \mathbb{R}^d \) and \( \lambda \in [0,1] \). There are essentially two possibilities:
\begin{itemize}
    \item If \( x,x' \in S \), then $\lambda x + (1-\lambda)x' \in S$ by convexity, and looking at the definition of $\hat f$ and using the convexity of $f$, we have:
     \begin{align*}\hat{f}(\lambda x + (1-\lambda)x') &= f(\lambda x + (1-\lambda)x')\\
       &\leq \lambda f(x) + (1-\lambda) f(x')\\
        &= \lambda \hat f(x) + (1-\lambda)\hat f(x'). \end{align*}
    \item If, say, \( x \notin S \), then \( \lambda \hat{f}(x) + (1-\lambda)\hat{f}(x') =+\infty \), and the condition for the convexity of $\hat f$ is trivially verified, since $\hat f(z) \leq +\infty$ for all $z$. The same happens if \( x' \notin S \).
\end{itemize}

In any case, the line segment joining any two points on the graph of $\hat f$ lies above the graph between those two points, so $\hat f$ is convex.\QED\\

\textit{Luis Antonio: 50\%}, \ \textit{Antonio: 50\%}\\
\textbf{Exercise 5. }\emph{If \( Q \) is a symmetric, positive definite \( d \times d \) matrix, show that \( f(x) = x^TQx, \ x \in \mathbb{R}^d \), is a convex function.}\\

Let $x,x'\in \mathbb R^d$ and $\lambda\in[0,1]$. We want to show that
\[
f(\lambda x + (1-\lambda )x')\leq \lambda f(x)+(1-\lambda)f(x'),
\]

or equivalently, that
\[
f(\lambda x + (1-\lambda )x') - \lambda f(x)+(1-\lambda)f(x') \leq 0.
\]

The result follows after some elementary algebraic manipulations:
\[
f(\lambda x + (1-\lambda )x') - \lambda f(x)-(1-\lambda)f(x') =
\]
\[
=\lambda^2x^TQx + 2\lambda(1-\lambda)(x')^TQx + (1-\lambda)^2 (x')^TQx' -\lambda x^TQx - (1-\lambda)(x')^TQx' =
\]
\[
= \lambda(\lambda - 1)\left[x^TQx + (x')^TQx' - 2x^TQx' \right] = \underbrace{\lambda(\lambda - 1)}_{\leq 0}\underbrace{(x-x')^TQ(x-x')}_{> 0} \leq 0,
\]

where in the last inequality we have used that $Q$ is positive definite. Throughout the calculations we have also used that, since $Q$ is symmetric, $(x')^TQx=x^TQx'$.\QED\\

% %\emph{Solution. } Using Taylor's series, we know that \( f(x + h) = f(x) + \nabla f(x)h + o(h) \), then
% \[
%     % \begin{aligned}
%      f(x + h) &= (x+h)^TQ(x+h)= x^TQx + h^TQx +x^TQh + h^TQh\\
%      &= f(x) + x^T(Q + Q^T)h + h^TQh
%     \end{aligned}
% \]
% Where \( \|h^TQh\| \leq \|Q\|\|h\|^2 = o(h) \). Therefore \( \nabla f(x) = x^T(Q + Q^T) = 2x^TQ \). Reusing the same argument,
% \[
%      \nabla f (x+h) = 2x^T2 + 2h^TQ = f(x) + 2h^TQ \implies H(f)(x) = 2Q > 0.
% \]
%
% We know use that \( f \) is convex given that its Hessian its semi-definite positive.

\textit{Luis Antonio: 50\%}, \ \textit{Antonio: 50\%}\\
\textbf{Exercise 6. }\emph{Given a quadratic form \( q(w) = w^TQw + bw + c \), with \( Q \) a symmetric \( d\times d \) matrix, \( w,b\) being \(d\times 1 \) vectors and \( c \) a real number, derive its gradient and Hessian}.

\emph{Solution.}
Starting from the expanded form
\[
  q(w) = \sum_{i=1}^{d}\sum_{j=1}^d Q_{ij} w_iw_j + \sum_{i=1}^db_i w_i + c
\]
and taking into account that $Q_{ij}=Q_{ji}$, we can easily compute the partial derivatives
\[
    \frac{\partial q(w)}{\partial w_i} = 2Q_{ii}w_i + \sum_{\substack{j=1\\j\neq i}}^d w_j(Q_{ij}+Q_{ji}) + b_i = \sum_{j=1}^d 2Q_{ij}w_j +  b_i, \quad i=1,\dots, d,
\]

and
\[
\frac{\partial ^2 q(w)}{\partial w_i \partial w_j} = 2Q_{ij}, \quad i,j=1,\dots, d.
\]
As a result, we have
\[
     \nabla q(w) = \left(\frac{\partial q(w)}{\partial w_1},\dots, \frac{\partial q(w)}{\partial w_d}\right)^T = 2\left(\sum_{j=1}^d Q_{1j}w_j, \dots, \sum_{j=1}^d Q_{dj}w_j   \right)^T + (b_1,\dots, b_d)^T = 2Qw + b,
\]

and
\[
     Hq(w) =
     \begin{pmatrix}
        \dfrac{\partial ^2 q(w)}{\partial w_1 \partial w_1} & \cdots & \dfrac{\partial ^2 q(w)}{\partial w_1 \partial w_d} \\
        \vdots & & \vdots \\
        \dfrac{\partial ^2 q(w)}{\partial w_d \partial w_1} & \cdots & \dfrac{\partial ^2 q(w)}{\partial w_d \partial w_d} \\
     \end{pmatrix} = \begin{pmatrix}
         2Q_{11} & \cdots & 2Q_{1d}\\
         \vdots & & \vdots \\
         2Q_{d1} & \cdots & 2Q_{dd}
     \end{pmatrix} = 2Q.
\]\QED

\vspace{1em}

\textit{Luis Antonio: 50\%}, \ \textit{Antonio: 50\%}\\
\textbf{Exercise 7. }\emph{Let \( f:\mathbb{R}^d \to \mathbb{R} \)  be a function and assume that \( epi(f) \subset \mathbb{R}^d \times \mathbb{R} \)  is convex. Prove that then \( f \)  is convex}.

\emph{Solution.} Recall that in this case
\[
     \operatorname{epi}(f) = \{(x,t) \in  \mathbb R^d\times \mathbb R: \ t \geq f(x)\}
\]
is the graph above \( f \). Let $x,x'\in \mathbb R^d$ and consider \( a = (x, f(x)) \) and \( b = (x', f(x')) \), observing that both of them are in \( \operatorname{epi}(f) \). Then, since $\operatorname{epi}(f)$ is convex, for $\lambda \in [0,1]$ we have that $\lambda a + (1-\lambda)b \in \operatorname{epi}(f)$, which in turn implies that
\[
(\lambda x + (1-\lambda)x', \, \lambda f(x) + (1-\lambda)f(x')) \in \operatorname{epi}(f).
\]

Therefore, by definition of the epigraph, \( f(\lambda x + (1-\lambda)x') \leq \lambda f(x) + (1-\lambda)f(x')\), and hence $f$ is convex.\QED\\

\textit{Luis Antonio: 50\%}, \ \textit{Antonio: 50\%}\\
\textbf{Exercise 8.} \emph{Let \( f : \mathbb{R}^d \to \mathbb{R} \) be a convex function. Prove that \( \operatorname{epi}(f) \) is a closed set and that \( (x, f(x)) \in \partial \operatorname{epi}(f) \)  }.

\emph{Solution.} Let us show first that \( \operatorname{epi}(f)  \) is closed. There are two different approaches:

\begin{itemize}
  \item We could use the well-known fact that convex functions on open sets are continuous, and the characterization of closed sets as sets that contain all the limiting points of sequences in them. Indeed, if $\{(x_n, t_n)\}\subset \operatorname{epi}(f)$ with $(x_n, t_n)\to (x, t)$, we have:
  \[
  f(x_n) \leq t_n \implies f(x)=\lim_{n\to\infty} f(x_n) \leq \lim_{\n\to\infty} t_n = t \implies (x, t) \in \operatorname{epi}(f),
  \]
  where the limit can be taken inside the function argument by continuity.
\item Alternatively, we could show that \( \operatorname{epi}(f)  = \operatorname{cl}(\operatorname{epi}(f) ) \). To prove the non-trivial inclusion, let \( (x,t) \in \operatorname{cl}(\operatorname{epi}(f) ) \), which means that
\begin{equation}
  \label{eq:8}
\operatorname{epi}(f) \cap B((x,t), \delta) \neq \emptyset \ \ \forall \delta > 0.
\end{equation}

Suppose by contradiction that $(x, t) \notin \operatorname{epi}(f)$ and observe that since $f$ is convex, $\operatorname{epi}(f)$ is also convex (the proof is straightforward and similar to that of Exercise 7). Therefore, by the Projection Theorem\footnote{We only need the existence part, which follows directly from Weierstrass' Theorem applied to the continuous function $z\mapsto \lVert z - x \rVert$ on the compact set $\{x' \in \operatorname{cl}(\operatorname{epi}(f)): \ \lVert x' - x \rVert \leq \lVert z - x \rVert\}$ for a fixed $z\in \operatorname{epi}(f)$.} a minimum distance $\alpha > 0$ from $(x, t)$ to $\operatorname{epi}(f)$ is attained, and so it holds that
\[
\operatorname{epi}(f) \cap B((x, t), \alpha/2) = \emptyset,
\]
contradicting \eqref{eq:8}.
\end{itemize}

Now, it is obvious that $(x, f(x))\in \operatorname{epi}(f)=\operatorname{cl}(\operatorname{epi}(f) )$, since $f(x)\geq f(x)$. To show that $(x, f(x))\in \partial \operatorname{epi}(f)$, it suffices to show that it doesn't belong to $\operatorname{int}(\operatorname{epi}(f))$, since
\[
\partial \operatorname{epi}(f) =\operatorname{cl}(\operatorname{epi}(f) ) - \operatorname{int}(\operatorname{epi}(f)).
\]

Indeed, if it weren't the case we could find a $\delta > 0$ such that
\[
B((x, f(x)), \delta) \subset \operatorname{epi}(f),
\]

so it would hold that, for example, $(x, f(x)-\delta/2) \in \operatorname{epi}(f)$. But by definition of the epigraph, this means that $f(x)\leq f(x) - \delta/2$, which is a contradiction given that $\delta>0$.\QED\\



\textit{Luis Antonio: 50\%}, \ \textit{Antonio: 50\%}\\
\textbf{Exercise 9. }\emph{Prove that if \( f \)  is strictly convex, it has a unique global minimum}.

\emph{Solution}. We will prove that for strictly convex functions defined on convex sets, there can be at most one global minimum. Note that it is still possible for a strictly convex function to have no minima (consider the function $x\mapsto e^x$ on $\mathbb R$).

Let $S\subset \mathbb R^d$ be convex and $f:S \to \mathbb R$ be strictly convex. We argue by contradiction and suppose $x^*, \tilde x \in S$ are two \textit{distinct} global minimums of $f$, that is,
\[
x^\ast \neq \tilde x \quad \text{and} \quad f(x^*)=f(\tilde x) \leq f(x) \quad \forall x \in S.
\]

Now, since $f$ is strictly convex, for any $\lambda \in (0,1)$ we have
\[
f(\lambda x^* + (1-\lambda)\tilde x) < \lambda f(x^*) + (1-\lambda)f(\tilde x) = f(x^\ast),
\]

which contradicts the fact that $x^*$ is a global minimum of $f$. Therefore, $f$ can have at most one global minimum.\QED\\

\textit{Luis Antonio: 50\%}, \ \textit{Antonio: 50\%}\\
\textbf{Exercise 10. }\emph{Let \( f,g: S \subset \mathbb{R}^d \to \mathbb{R} \)  be two convex functions on the convex set \( S \) . Prove that, as subsets, \( \partial(f+g)(x) \subset \partial f(x) + \partial g(x) \) for any $x\in S$. }

\emph{Solution}. This is (part of) the statement of the Moreau-Rockafellar Theorem that we saw in class. This is the hard part, since the other inclusion can be easily proved, and we do so here for completeness.\newpage

Fix $x\in S$ and let $c\in \partial f(x) + \partial g(x)$. We can decompose it as $c=c_1+c_2$ with $c_1\in\partial f(x)$ and $c_2\in \partial g(x)$. Then, by definition of subgradient and the distributive property of the scalar product, we have
\begin{align*}
\forall x' \in S \quad (f+g)(x') &= f(x') + g(x')\\
&\geq f(x) + g(x) + c_1(x'-x) + c_2(x'-x)\\
&= (f+g)(x) + c(x'-x),
\end{align*}
so that $c\in \partial(f+g)(x)$. The proof of the other inclusion can be consulted in the book \textit{Convex Analysis} by Rockafellar himself\footnote{Theorem 23.8 in Rockafellar, R. T. (1997). \textit{Convex analysis} (Vol. 36), p. 223. Princeton University Press.}, in which ...

Prior to consulting any references we tried to prove the theorem by ourselves, and came up with the following incomplete proof. ......\\

\textit{Luis Antonio: 50\%}, \ \textit{Antonio: 50\%}\\
\textbf{Exercise 11. }\emph{Compute the proximal of \( f(x) = 0 \) and of \( g(x) = \frac{1}{2}\|x\|^2 \).}

\emph{Solution.} Recall that the proximal operator can be expressed as
\[ \operatorname{prox}_f(x) = \arg\min_z\left\{f(z) + \frac{1}{2} \|z - x\|^2 \right\}. \]

The strategy will be to compute and minimize this expression in each case.
\begin{itemize}
    \item On the one hand, we have \( \operatorname{prox}_f(x) = \arg\min_z\left\{ \frac{1}{2} \|z - x\|^2 \right\} \), but \( \|z-x\|^2 \) is clearly minimized at \( z = x \), so $\operatorname{prox}_f(x)=x$ for all $x$.
    \item  On the other hand, \( \operatorname{prox}_g(x) = \arg\min_z\{\frac{1}{2}\|z\|^2 + \frac{1}{2} \|z - x\|^2 \}\). We know that for differentiable functions the minimizers obey the first order condition $\nabla =0$, so:
    \[
         0=\nabla_z \left(\frac{1}{2} \|z\|^2 + \frac{1}{2}\|z-x\|^2\right) = z + z - x\implies z = \frac{1}{2}x.
    \]
    The point $x/2$ is indeed a minimizer of our function in $z$, for example because its Hessian matrix is $2I$, which is positive definite. Therefore, \( \operatorname{prox}_g(x) = x/2 \) for all $x$.\\
\end{itemize}

\textit{Luis Antonio: 50\%}, \ \textit{Antonio: 50\%}\\
\textbf{Exercise 12. }\emph{Assume that \( f \)  is convex. Prove that for any \( \lambda > 0, \partial(\lambda f)(x) = \lambda \partial f(x) \) as subsets.}

\emph{Solution}. Instead of showing the double inclusion, we prove the equivalence directly. We have:
\[
    \begin{align*}
     \partial(\lambda f)(x) &= \{c \in \mathbb{R}^d: \ \lambda f(x') \geq \lambda f(x) + c(x' - x)\ \forall x' \in S\} \quad &\text{[Definition]}\\
     &= \{c \in \mathbb{R}^d \ : \  f(x') \geq f(x) + \lambda^{-1} c(x' - x)\ \forall x' \in S\}\quad &\text{[Divide by $\lambda>0$]} \\
     &= \{\lambda c \in \mathbb{R}^d \ : \  f(x') \geq f(x) + c(x' - x)\ \forall x' \in S\}\quad &\text{[Rename]} \\
     &= \lambda \partial f(x).\quad &\text{[Definition]}
   \end{align*}
\]
\QED

\vspace{.5em}

\textit{Luis Antonio: 50\%}, \ \textit{Antonio: 50\%}\\
\textbf{Exercise 13. }\emph{Compute the proximals of the hinge \( f(x) = max\{0, -x\}  \) and the \( \epsilon \)-insensitive \( g(x)=max\{0, |x| - \epsilon\} \) loss functions}.

\emph{Solution.} We proceed as in Exercise 11.
\begin{itemize}
    \item On the one hand, we have \( \operatorname{prox}_f(x) = \arg\min_z\{\max\{0, -z\} + \frac{1}{2} (z - x)^2 \} = \arg\min_z h(z) \). We are looking for the minimizer of the function
        \[
             h(z) = \begin{cases}
                 \frac{1}{2} (z - x)^2, &\text{ if } z \geq 0,\\
                -z + \frac{1}{2} (z - x)^2, &\text{ if } z \leq 0.\\
             \end{cases}
        \]
        If the minimizer is attained at $z> 0$, then clearly $z=x$, meaning that $\operatorname{prox}_f(x)=x$ for $x\geq 0$. If it is attained at $z<0$, we have:
        \[
        0=h'(z)= -1 + z - x \implies z = x+1,
        \]
        which again is easily shown to be a genuine minimum ($h''>0$). This implies that $\operatorname{prox}_f(x)=x+1$ for $x< -1$. The value of the proximal at the remaining values must necessarily be the only point of non-differentiability of $h$, namely, $0$. Thus, we have shown that
        \[
             \operatorname{prox}_f(x)= \begin{cases}
             x+1, & \text{ if } x\leq -1,\\
               0, & \text{ if } -1 \leq x \leq 0,\\
              x, & \text{ if } x \geq 0.
           \end{cases}
        \]
    \item On the other hand, \( \operatorname{prox}_g(x) = \arg\min_z\{\max\{0, |z| -\epsilon\} + \frac{1}{2} (z - x)^2 \} = \arg\min_z h(z) \). In this case, we seek the minimizer of
        \[
             h(z)= \begin{cases}
                \frac{1}{2} (z - x)^2, &\text{ if } |z| \leq \epsilon,\\
                z - \epsilon + \frac{1}{2} (z - x)^2, &\text{ if }   z \geq \epsilon,\\
                -z - \epsilon + \frac{1}{2} (z - x)^2, &\text{ if }   z \leq -\epsilon.
             \end{cases}
        \]

Following the same reasoning as before, it is immediate to see that $\operatorname{prox}_g(x)=x$ for $|x|<\epsilon$. If the minimizer is attained when $z>\epsilon$, we have $0=h'(z)=1+z-x$, so $z=x-1$, and $\operatorname{prox}_g(x)=x-1$ for $x>1+\epsilon$. Similarly, $\operatorname{prox}_g(x)=x+1$ for $x< -\epsilon-1$.

For the points of non-differentiability, observe that if $\epsilon<x<1+\epsilon$ the minimizer is attained at $z=\epsilon$, while for $-1-\epsilon<x<\epsilon$ it is attained at $z=-\epsilon$ (a simple check suffices to convince us). Thus, we have

        \[
             \operatorname{prox}_f(x)= \begin{cases}
               x+1, & \text{ if } x< - 1 - \epsilon,\\
               -\epsilon,  & \text{ if } -1-\epsilon<x<\epsilon,\\
               x, & \text{ if } |x| < \epsilon,\\
               \epsilon,  & \text{ if } \epsilon < x < 1+\epsilon,\\
               x-1, & \text{ if } x>1+\epsilon.
           \end{cases}
        \]
        \QED
\end{itemize}

\textbf{Exercise 14. }\emph{If \( (p_1,\dots,p_n) \) is a probability distribution, prove that its entropy \( H(p_1, \dots, p_n) = - \sum_{i=1}^n p_i \log p_i\) is a concave function. Show also that its maximum is \( \log n \), attained when \( p_i = \frac{1}{n}\) for all $i$}.

\emph{Solution. } In the first place, we know that $H$ is twice-differentiable on the open set $(0,1)^n$. Moreover, since
\[
\frac{\partial H}{\partial p_i} = -\log p_i - 1 \quad \text{and} \quad \frac{\partial^2 H}{\partial p_i\partial p_j} = \frac{-1}{p_i}\delta_{ij},
\]
we have
\[
\operatorname{Hess}(H) = \operatorname{diag}\left\{\frac{-1}{p_i}\right\}_{i=1,\dots,n} \prec 0.
\]

Since the Hessian is negative definite, $H$ is concave. To solve the constrained optimization problem, we consider the corresponding Lagrangian
\[ L(p_1,\dots, p_n; \mu) = - \sum_{i=1}^n p_i \log p_i + \mu\left(\sum_{i=1}^n p_i - 1\right),\]

where the restriction $\sum p_i=1$ appears because the $p_i$ are a probability distribution. Now, we have
\[
     \frac{\partial L}{\partial p_i} = -\log p_i - 1 + \mu, \ \  i=1,\dots,n \quad \text{and} \quad \frac{\partial L}{\partial \mu} =\sum_{i=1}^n p_i - 1.
\]

To find the points where all derivatives vanish simultaneously, we have to solve the system
\[\begin{cases}
    \log p_i &= \mu - 1, \quad i=1,\dots,n,\\
    \sum p_i &= 1.
\end{cases}\]

Looking at the first equation, since the right hand side doesn't depend on $i$, it holds that $p_i=p_j= e^{\mu - 1}\geq 0$ for all $i,j$, that is, all the \( p_i \) are equal. But since they must add up to $1$, the only possibility is that $p_i=1/n$ for all $i$. The restriction $p_i\geq 0$ is clearly satisfied, and since $H$ is concave the resulting value is indeed a maximum, whose value is
\[
H(1/n,\dots, 1/n) = -\frac{1}{n}\sum_{i=1}^n \log{\frac{1}{n}} = \frac{n\log n}{n}=\log n.
\]

\textbf{Exercise 15. }\emph{We have worked out the dual problem for the soft SVC problem. Do the same for the simpler \textbf{hard} SVC problem}
\[
\min_{w,b} \frac{1}{2}\|w\|^2
\]
\emph{subject to \( y^p(wx^p + b) \geq 1 \). What are here the KKT conditions? }




\end{document}
